\documentclass[a4paper]{article} %formato de plantilla que vamos a utilizar

\usepackage[utf8]{inputenc}
\usepackage[spanish]{babel}
\usepackage[margin=2cm, top=2cm, includefoot]{geometry} % Sirve para la geometria de espacio 
\usepackage{graphicx} % para la insercion de imagenes
\usepackage[table,xcdraw]{xcolor} % Para la deteccion de colores url (txcolorbox )
\usepackage[most]{tcolorbox} % Para la insercion de cuadros en la portada
\usepackage{fancyhdr} % Definir el estilo de la pagina
\usepackage[hidelinks]{hyperref} % integracción de hinpervinculo
\usepackage{parskip} % Arreglo de la tabulacion del informe
\usepackage[figurename=paper]{caption} % renombrar el detalle de la imagen
\usepackage{smartdiagram}
% Declaracion de colores
\definecolor{grenPortada}{HTML}{69AB4F}

% Declaracion de variables
\newcommand{\portada}{index.png}
\newcommand{\empresalogo}{security.jpeg}
\newcommand{\Maquina}{paper} % Nombre de la maquina
\newcommand{\logoMaquina}{paper.png} % Logo de la maquina
\newcommand{\starDate}{07 de Marzo 2022}

% adicionales
\setlength{\headheight}{48.2pt}
\pagestyle{fancy}
\fancyhf{}
\lhead{\includegraphics[width=6cm]{\empresalogo}}\rhead{\includegraphics[height=1.5cm]{\logoMaquina}}
\renewcommand{\headrulewidth}{3pt}
\renewcommand{\headrule}{\hbox to\headwidth{\color{grenPortada}\leaders\hrule height \headrulewidth\hfill}}
% Comienza del documento

\begin{document}

% creacion de portada
    \begin{titlepage}
        % Saltado de pagina
        \cfoot{\thepage}
        \centering
        \includegraphics[width=0.6\textwidth]{\portada}\par\vspace{1cm}
        {\scshape\LARGE \textbf{Informe Técnico}\par}
        \vspace{0.2cm}
        {\Huge\bfseries\textcolor{grenPortada}{Maquina \Maquina}\par} % variable de maquina
        \vfill\vfill
        \includegraphics[width=\textwidth,height=10cm,keepaspectratio]{\logoMaquina}\par\vspace{1cm}
        \vfill
        \begin{tcolorbox}[colback=red!5!white,colframe=red!75!black]
            \centering
            Este documento es confidencial y contiene informacion sensible.\\No deberia ser ingreso o compartido con tercero
        \end{tcolorbox} 
        \vfill\vfill\vfill\vfill % dar espacios
        {\large \starDate\par}
          
    \end{titlepage}
    % <-Terminacion de portada->

    % creacion de indice
    \clearpage
    \tableofcontents
    \clearpage
    % <----------------------Terminacion------------------------------->
    
    % <-------Creacion de cabeceras----->
    \section{Detalles}
    El presente informe detalla los resultados obtenidos y encontrados en la realizacion de auditoria a la máquina {\textbf\Maquina} de la plataforma \href{https://hackthebox.eu}{\text{\color{blue}Hackthebox}}.
    \vspace{1.2cm}
    \begin{figure}[h]
        \centering
        \includegraphics[width=\textwidth]{imagen/ip.png}
        \caption{IP de la máquina}
        
    \end{figure}
    \vspace{2.0cm}
    \section{Objetivos}
    El objetivo general de la auditoria es presentar los diversos fallos y vulnerabilidades del sistema \textbf{\Maquina}, Y la facilidad de como un atacante
    puede acceder y robar la informacion del sistema, darñar la infraestructura critica de los sistemas infromaticos con tecnicas de explotación y vectores de ataques dirigidos.
    \vspace{0.2cm}
    \subsection{Concideraciónes}
    Una vez finalizada la auditoria se llevara a cabo una fase de concientizacion de usuarios, para hacerle saber a los trabajadores
    las buenas practicas de seguridad y llevar a cabo la cumplementura de politicas de privacidad y de cumpliento en la empresa
    \vspace{0.2cm}

    \begin{figure}[h]
        \begin{center}
            \smartdiagram[priority descriptive diagram]{
                Reconocimiento del sistema,
                Deteccion de fallos y vulnerabilidades,
                Explotacion de vulnerabilidades,
                Protección del sistema
            }
        \end{center}
        \caption{Ezquematización}
        
    \end{figure}
    \vspace{0.5cm}

    \subsection{Disposiciónes}
    Las disposiciones establecidas en la auditoria marcan una referencia rapida y eficiencia en la maquina auditada para una prueba de evaluacion rapida y simulacion de ataques a la infraestructura de la plataforma. 
    \vspace{1.0cm}
    \section{Alcance}
    Se establecio un alcance para la auditoria de 2 mes para no dañar la confidencial, integridad y disponibilidad de la infraestructura
    \subsection{Accesos}
    Se otorgo acceso a un dominio en el cual se puede realizar la primera etapa de reconomiento para un mayor tiempo factible
    \vspace{0.2cm}
    \begin{figure}[h]
        \begin{center}
            \smartdiagram[priority descriptive diagram]{
                Datos de Dominio,
                informacion de puertos,
                Informacion de servicios    
            }
        \end{center}
        \caption{Diagrama}
    \end{figure}
   
    \vspace{16.5cm}
    \section{\textbf{Reconocimiento de vulnerabilidades}}
    \vspace{0.3cm}
    \subsection{Escaneo de Dominio}
    En la identificacion y escaneo de direccion IP de la maquina  \textbf{\Maquina} se identificaron varios puertos abiertos. Puertos comunes (80, 22) donde se encontro 
    un servidor Apache[2.4.37]
    \vspace{0.5cm}
    \begin{figure}[h]
        \centering
        \includegraphics[width=0.9\textwidth]{imagen/nmap.png}
        \caption{Reconocimiento de dominio}
    \end{figure}
    \subsection{Reconocimiento de servicios}
    Se realizo un escaneo del dominio intenso, para identificar la versión de los puertos y sistema operativo
    y se identifico que la máquina \textbf{\Maquina} cuenta con versión desactualizadas pero no explotables
\begin{figure}[h]
    \begin{center}
        \includegraphics[width=0.9\textwidth]{imagen/scan_servicios.png}
    \end{center}
    
\end{figure}
\vspace{0.5cm}
    Una vez localizado los puertos y servicios a traves de la herramienta \textbf{nmap}, que se estan ejecutando en la maquina \textbf{\Maquina} nos enfocamos a nivel web 
logrando encontrar ya mencionado el servidor activo en el puerto 80, identificando la siguiente pagina.
\vspace{0.5cm}
\begin{figure}[h]
    \centering
    \includegraphics[width=0.8\textwidth]{imagen/apache.png}
    \caption{Servidor web}
    \vspace{0.5cm}  
\end{figure}
\subsection{Reconocimiento de Subdominios}
Con la herramienta \textbf{Gobuster} logramos encontrar un subdominios llamado \textbf{Office.paper} donde logramos acceder al apartado donde se encontro una vulnerabilidad en dicha pagina.
\vspace{0.5cm}
\begin{figure}[h]
    \centering
    \includegraphics[width=0.6\textwidth]{imagen/gobuster_paper.png} 
    \caption{Gobuster scan}

\end{figure}

\clearpage
Dentro del subdominio encontrado por la herramienta pudismos acceder libremente y auditar el codigo directo de la pagina,
sin encontrar nada relevante.
\vspace{0.5cm}
\begin{figure}[h]
    \centering
    \includegraphics[width=\textwidth]{imagen/office.paper.png}
    \caption{Office.paper.htb}
    
\end{figure}
\section{Analisis de Vulnerabilidades}
\vspace{0.1cm}
\textbf{Subdomain Command Exploit}
Luego de la identificación de la pagina se logro encontrar un exploit que nos permitia identificar un segundo subdominio
en la pagina de \textbf{office.paper.htb}. Utilizando el exploit de wordpress. \href{https://wpscan.com/vulnerability/3413b879-785f-4c9f-aa8a-5a4a1d5e0ba2}{\textbf{\color{blue}{Exploit}}}.
\vspace{0.5cm}
\begin{figure}[h]
    \centering
    \includegraphics[width=0.7\textwidth]{imagen/chat.paper.png}
    \caption{Subdomain chat.office.paper}    
\end{figure}
\clearpage
\subsection{Remote Command}
Podemos leer archivos y ver directorios  a traves del usuario \textbf{recyclops}.
\vspace{0.2cm}
\begin{figure}[h]
    \centering
    \includegraphics[width=\textwidth]{imagen/recyclops.png}
    \caption{local File Inclusión}
    
\end{figure}
\clearpage
\subsection{Remote File Inclusión}
En ese chat en particular podemos ejecutar codigo remoto para lograr ver dentro de los archivos   \textbf{/hubot/.env}.
\vspace{0.2cm}
\begin{figure}[h]
    \centering
    \includegraphics[width=\textwidth]{imagen/user.recyclops.png}
    \caption{Remote File Inclusion}
    
\end{figure}
\clearpage
podemos observar que dentro \textbf{/hubot/.env} se pueden visualizar usuarios y credenciales de la maquina \textbf{\Maquina}.
A si mismo ganando acceso a través del puerto 22 \textbf{SSH} con las credenciales obtenidas a traves de la plataforma.

\section{Explotacción de Vulnerabilidades}

Con las credenciales obtenidas en la etapa anterior podemos iniciar directamente en el servidor y meternos en el sistema principal.
\vspace{0.5cm}
\begin{figure}[h]
    \centering
    \includegraphics[width=\textwidth]{imagen/ssh.png}
    \caption{Servidor Inicial SSH}  
\end{figure}
\vspace{0.5cm} 

Dentro del servidor \textbf{SSH} encontramos un archivos \textbf{user.txt} al visualizarlo con cat encontramos la flag
que se nos pide en la plataforma de \href{https://hackthebox.eu}{\text}{\color {blue}{hackthebox}}.
\vspace{0.2cm}
\clearpage
\subsection{Escalación de privilegiós}
Para la escalación de privilegios en la maquina \textbf{\Maquina} encontramos un \href{https://github.com/Almorabea/Polkit-exploit/blob/main/CVE-2021-3560.py}{\textbf{\color{blue}{Exploit}}}
\begin{figure}[h]
    \centering
    \includegraphics[width=\textwidth]{imagen/exploit_escalation.png}
    \caption{Escalación}
    \label{fig:servicesResults}    
\end{figure}

\vspace{0.2cm}
Podemos ver el codigo del exploit como si visualiza en la figura \ref{fig:servicesResults} este crea un usuario
con permisos de root.
\clearpage
Ya como usuario root encontramos en el directorio \textbf{/root/} la flag restante para completar el desafio.
\begin{figure}[h]
    \centering
    \includegraphics[width=0.8\textwidth]{imagen/root.png}
    \caption{Flag de root}
    
\end{figure}

\section{Borrado de Registros}
Realizamos diferentes tecnicasde borrado y sobre escritura para eliminar cualquier resgistro
dejado en la maquina \textbf{\Maquina}.
\vspace{0.3cm}
\subsection{Lectura de resgistro logs}
Ya obtenido las flag por ultimo paso es eliminar todo aquello que podismos a ver dejado en el sistema.
Eliminamos los registros logs para pasar desapersivido para cualquier sysadmin.
\vspace{0.3cm}
\begin{figure}[h]
    \centering
    \includegraphics[width=0.7\textwidth]{imagen/log1.png}
    \caption{Eliminación de logs}
    \label{fig:servicesLogs}   
\end{figure}

\vspace{0.5cm}
Cómo podemos ver en la figura \ref{fig:servicesLogs} hay muchos archivos logs que registran informacion del sistema.
Cómo la lista de maquinas que se conectan al sistema y que comandos y archivos van visualizando.
\clearpage
\subsection{Eliminación de registros logs}
\vspace{0.3cm}
Lo importante es borrar todo aquel registro que nos comprometa como atacantes, para esto seleccionamos algunos registros importantes.
Estos son \textbf{lastatlog, auth.log, cron}etc.
\vspace{0.3cm}
\begin{figure}[h]
    \centering
    \includegraphics[width=0.8\textwidth]{imagen/lastat.png}
    \caption{Eliminacion de registro log \textbf{Lastatlog}}
    \vspace{0.3cm}
    \includegraphics[width=0.8\textwidth]{imagen/audit_log.png}
    \caption{Eliminación de registro log auth}   
\end{figure}
\clearpage
\subsection{Sobreescritura de archivos}
\vspace{0.3cm}
Borramos los registros ya mencionados y los volvemos a escribir con los datos siguientes
\vspace{0.5cm}
\begin{figure}[h]
    \centering
    \includegraphics[width=\textwidth]{imagen/sobreescritura.png}
    \caption{Sobreescritura de registros logs}
    \label{fig:serviceseescritura}    
\end{figure}
\vspace{0.3cm}
La manera mas rapida pero poco segura de eliminar cualquier rastro es sobreescribir un archivo eliminado con el mismo nombre, así como
se ve en la figura \ref{fig:serviceseescritura}.
\end{document}